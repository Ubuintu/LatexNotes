% Title page
% Synatx: \commandname{option}
\documentclass{article}

% Preamble; here we set up the values for the \maketitle command
% ** NEVER USE '&' INSTEAD OF 'and' **
\title{Linux Notes}
\date{10-10-2022}
\author{Tommy Bui}

\begin{document}
	\maketitle
	\newpage
	\pagenumbering{arabic}
	
	\section{High-level View of Unix Environment}
	\begin{itemize}
		\item To best understand how an operating system works, keep in mind the concept of abstraction.
		\begin{itemize}
			\item Abstraction focuses on the basic purpose and operation of an object.
		\end{itemize}
		% Example of bolded text
		\item There are many times for an abstracted subdivision in software. In these notes, the term {\bf component} 
		\item This chapter provides a high-level overview of the components that make up a Linux system.
	\end{itemize}

	\subsection{Levels and Layers of Abstraction in Linux}
	\begin{itemize}
		
		\item Abstraction helps break down the Linux operating system into easy-to-understand components.
		
		\item We arrange components into layers or levels, classifications or groupings of components according to where the components lay between the user and hardware.
		
		\begin{itemize}
			\item i.e. Web browsers, games, etc. are at the top layer; the bottom layer consists of the memory in hardware which is composed of 0's and 1's.
		\end{itemize}
	
		\item A Linux OS consists of 3 main levels:
		
		\begin{itemize}
			
			\item The base consists of hardware:
				\begin{itemize}
					\item Hardware includes the memory as well as the Central Processing Unit(s) to perform computation or RD/WR to memory.
					\item Devices such as disks and network interfaces are also part of the hardware.
					\item Examples of Hardware: CPU, main memory (RAM), Disks, Network ports, etc.
				\end{itemize}
			
			\item The next level up is the kernel:
				\begin{itemize}
					\item The kernal is consider the software within memory that tells the CPU where to look for its next task.
					\item As a mediator, the kernal manages hardware (i.e. main memory) and is the primary interface between hardware and any running program.
					\item Linux Kernal contains: System calls, Process Management, Memory Management, and Device Drivers
				\end{itemize}
			
			\item Processes:
				\begin{itemize}
					\item Running programs that are managed by the kernal, make up the system's upper level known as {\bf user space} (i.e. all web servers run as {\bf user processes}).
					\item User Processes include GUI, Servers and Shell
				\end{itemize}
			
			\item The main difference between how the kernal and the user processes run is that the \underline{kernal runs in kernal mode} and \underline{user processes run in user mode}
			
			\item Code running in kernal mode has unrestricted access to the processor and main memory. This can be powerful but is a dangerous privilege that can cause the kernal to easily corrupt and crash the entire system. 
			
			\item Memory area that the \underline{only the kernal can access is {\bf kernal space}}
			
			\item Unlike kernal mode, user mode is restricted to a subset of memory and safe CPU operations
			
			\begin{itemize}
				\item The Linux kernal can run kernal threads, which are similar to processes but have access to kernal space (i.e. kthreadd and kblockd)
			\end{itemize}
			
			\item \textit{User space} refers to the parts of the main memory that user processes can access. If a process were to crash, the consequences are limited and can be repaired by the kernel
			
				\begin{itemize}
					\item i.e. If your web browser crashes, it won't stop the scientific computation background processes that has been running for days
					\item In theory, a user process gone haywire can't damage the majority of the system. However, user processes may affect other parts of your system
					\item i.e. With the correct permissions, a user process can damage data on a disk
				\end{itemize}
				
		\end{itemize}
	\end{itemize}

	\subsection{Hardware: Understanding Main Memory}
	
\end{document}